\document[./main_hagedorn.tex]{subfiles}
\usepackage{amsmath}
\newtheorem{theorem}{Theorem}[section]
\newtheorem{corollary}{Corollary}[theorem]
\newtheorem{lemma}[theorem]{Lemma}

\begin{document}
%The algorithms outlined in   
%\cite{betzNonadiabaticTransitionsMultiple2019} 
%into the 
%algorithm as outline in Chapter V of \cite{lubichQuantumClassicalMolecular2008}.
%In \cite{hagedornRaisingLoweringOperators1998}, Hagedorn considered a particular parametrisation 
%for the covariance matrix of a complex semi-classical Gaussian wavepacket. Such parametrization 
%lends itself to useful consequences for the solution of the Schrodinger equation. 
%\textcolor{red}{Comparing to the variational gaussian approximation, interpretation as 
%orthogonal and symplectic projection will point to the gain in this parametrization}
The superadiabatic formulas of equation  ... allow us to to effectively reduce a 
coupled systems of two PDEs to two uncoupled PDEs, within the error made by the approximation 
of the transmitted wavepacket. In this framework, the numerical challenge 
stems from solving the one level BO dynamics efficiently. 
In the literature there are various approaches for doing this, often revolving around 
Gaussian basis sets. \textcolor{red}{I actually do not know about any methods, this 
is something to look into}.
In this chapter we explore an approach which is rather unknown within the Chemistry community 
and absent from any QMD commercial or open-source software package. 
It consists in evolving a set of parametrised orthogonal basis functions, known as Hagedorn wavepackets. 
Precisely, these are the d-dimensional equivalent of Hermite polynomials, eigenfunctions of 
the Harmonic oscillator [reference].
When the potential is quadratic one has to evolve only the set of parameters according to 
a system of ODEs. 
For more general potentials, one accounts for the non-quadratic remainder term 
by applying the Dirac-Frenkel variational principle; this leads to a set of integrals 
involving the non-quadratic remainder term [reference].  
Our contribution consists in merging the Superadiabatic formula with 
this classical evolution of the Hagedorn wavepackets. 
In Section 1 we introduce the Hagedorn wavepackets and their properties as 
outlined in \cite{hagedornRaisingLoweringOperators1998}. In Section 2 
we address the projection of the transmitted wavepacket back onto an 
Hagedorn basis. The nature of these integrals is similar to the ones 
encountered in accounting the non-quadratic terms of the quadratic potential.
\end{document}
