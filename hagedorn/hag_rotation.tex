\document[./main_hagedorn.tex]{subfiles}
\newtheorem{theorem}{Theorem}[section]
\newtheorem{corollary}{Corollary}[theorem]
\newtheorem{lemma}[theorem]{Lemma}
\usepackage{amsmath}

\begin{document}
%%%%%%%%%%%
%
%
%   Outline and motivation
%
%
%%%%%%%%%%
\subsection{Change of co-ordinates}
In order to apply the one dimensional transition formula we need to rotate the co-ordinate 
axes such that one co-ordinate lies parallel to $p_{c}$.
Hence, we want a matrix $R$ satisfying
\begin{itemize}
  \item $Rp_c = (\|p_c\|,0,...,0)$ - align momentum vector with one axis 
  \item $R^T = R^{-1}$ (orthogonality)
  \item $\det(R) = 1$ (keep the same orientation of the basis)
\end{itemize}
We construct it by applying the Gram-Schmidt process to the vector 
$\frac{p_c}{\| p_c \|}$, i.e. we form an orthonormal basis 
$\{r_1 = \frac{p_c}{\| p_c \|}, r_2, ..., r_d \}$. Stacking this vectors into 
a matrix R as rows yields the desired rotation matrix $R$, that is 
\begin{align}
  R =
  \begin{bmatrix}
    \frac{p_c^T}{\| p_c \|} \\
    r_2^T \\
    ... \\
    r_d^T \\
  \end{bmatrix}
\end{align}
If $\det(R) = -1$ then multiply its last column by $-1$.
(\textcolor{red}{There is also another way of doing it called Householder reflection
which is cheaper although not sure in what it consists yet} 
\\
Applying the change of coordinates to the wavepacket 
at the crossing yields
\begin{align}
  \hat{\psi}^{\pm}(\tilde{\xi},t_c) 
  &= 
  \exp{\left(-\frac{i}{\epsilon}S^{\pm}(t_c)\right)}
  \sum_{k \in \mathcal{K}} c_k^{\pm}(t_c) 
  \varphi_k^\epsilon[\hat{\Pi}^{\pm}(t_c)](R^{T}\tilde{\xi})
\end{align}
Since the $\varphi_k^{\epsilon}$'s obey recurrence relation 
\eqref{hagedorn:hagedorn_wavepackets:recursive_relation} we only need to consider 
the change of coordinates for the zero$^{th}$ order wavepacket 
\eqref{hagedorn:hagedorn_wavepackets:complex_gaussian}
\begin{align}
  \varphi_0^{\epsilon}[\hat{\Pi}](\tilde{\xi}) 
  &
  \propto
  \exp{\left( \frac{i}{2\epsilon}
      (R^T\tilde{\xi}-p)^TQP^{-1}(R^T\tilde{\xi}-p) +
  \frac{i}{\epsilon}q^T(R^T\tilde{\xi}-p)\right)}
  \notag
  \\
  &=
  \exp{\left( \frac{i}{2\epsilon}
      (R^T(\tilde{\xi}-Rp))^TQP^{-1}R^T(\tilde{\xi}-Rp) +
  \frac{i}{\epsilon}q^TR^T(\tilde{\xi}-Rp)\right)}
  \notag
  \\
  &=
  \exp{\left( \frac{i}{2\epsilon}
      (\tilde{\xi}-Rp)^TRQP^{-1}R^T(\tilde{\xi}-Rp) +
  \frac{i}{\epsilon}(Rq)^T(\tilde{\xi}-Rp)\right)}
  \notag
  \\
  &=
  \exp{\left( \frac{i}{2\epsilon}
      (\tilde{\xi}-Rp)^TRQ(RP)^{-1}(\tilde{\xi}-Rp) +
  \frac{i}{\epsilon}(Rq)^T(\tilde{\xi}-Rp)\right)}
\end{align}
where we have simply applied transpose rules for each equality.
Perhaps expected, a rotation of the co-ordinate axes yields a new parameter
set $\hat{\Pi}^\prime = \left\{Rp, Rq, RP, RQ \right\}$ 
for the family of Hagedorn wavepackets. Indeed, the matrices $RP$ and 
$RQ$ still satisfy the symplectic properties of 
Lemma \ref{hagedorn:hagedorn_orthogonal_projection:lemma:symplectic_equations} since 
\begin{equation}
  \begin{split}
    & (RP)^T (RQ) - (RQ)^T(RP) =
    \\
    & P^TR^T RQ - Q^TR^TRP = 
    \\
    & P^TQ - Q^TP = 0
  \end{split}
\end{equation}
and similarly 
\begin{equation}
  \begin{split}
    & (RP)^* (RQ) - (RQ)^*(RP) =
    \\
    & P^*R^* RQ - Q^*R^*RP = 
    \\
    & P^*R^T RQ - Q^*R^TRP = 
    \\
    & P^*Q - Q^*P = 2iI_d
  \end{split}
\end{equation}
With these new set of parameters we can then use the recurrence relation of 
equation \eqref{hagedorn:hagedorn_wavepackets:recursive_relation} to yield 
the higher order wavepackets. 
%%%%%%%%
%
%
%       TWO DIMENSIONAL CASE
%
%
%%%%%%%%
    %In this case, the wavepacket at the crossing with the first component varying along the 
    %direction of $p_c$ is given by 
    %\begin{align}
    %  \hat{\psi}^{\pm}(\tilde{\xi},t_c) 
    %  &= 
    %  \exp{\left(-\frac{i}{\epsilon}S^{\pm}(t_c)\right)}
    %  \sum_{k \in \mathcal{K}} c_k^{\pm}(t_c) 
    %  \varphi_k^\epsilon[\hat{\Pi}^{\pm}(t_c)](R_{-\theta}\tilde{\xi})
    %  \\
    %  &
    %  \exp{\left(-\frac{i}{\epsilon}S^{\pm}(t_c)\right)}
    %  \sum_{k \in \mathcal{K}} c_k^{\pm}(t_c) 
    %  p_k(R_{-\theta}\tilde{\xi})\varphi_0^\epsilon[\hat{\Pi}^{\pm}(t_c)](R_{-\theta}\tilde{\xi})
    %\end{align}
    %where $\tilde{\xi} = R_{-\theta}\xi $  ??
\end{itemize}
\end{document}
