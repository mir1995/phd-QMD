\newpage
\textbf{Example. Two dimension}
\\
\textcolor{red}{Should definitely derive recursive relation to make sure it makes
sense. The following is unnecessary but should clear up any doubt. It could also be
written more coincisely but ... could let (x-q) but does not bother me too much}
\\
With $k \in \mathbb{N}^2$ we write out the recurrence relation explicitly
\begin{align}
  \begin{bmatrix}
    Q_{11} & Q_{12}
    \\
    Q_{21} & Q_{22}
  \end{bmatrix}
  \begin{bmatrix}
  \sqrt{k_{1} + 1}
  \varphi_{k + \langle 10 \rangle}
    \\
  \sqrt{k_{2} + 1}
  \varphi_{k + \langle 01 \rangle}
  \end{bmatrix}
  =&
  \sqrt{\frac{2}{\epsilon}}
  \begin{bmatrix}
    (x_1 - q_1)\varphi_k
    \\
    (x_2 - q_2)\varphi_k
  \end{bmatrix}
  -
  \begin{bmatrix}
    \overline{Q}_{11} & \overline{Q}_{12}
    \\
    \overline{Q}_{21} & \overline{Q}_{22}
  \end{bmatrix}
  \begin{bmatrix}
    \sqrt{k_1}\varphi_{k - \langle 10 \rangle}
    \\
    \sqrt{k_2}\varphi_{k - \langle 01 \rangle}
  \end{bmatrix}
\end{align}
and then try to obtain the integral representation that we
are after noting that only
$\varphi$ depends on $x$.
\begin{align}
  \begin{split}
  &\begin{bmatrix}
  Q_{11} \sqrt{k_{1} + 1} \varphi_{k + \langle 10 \rangle}
   +
   Q_{12} \sqrt{k_{2} + 1} \varphi_{k + \langle 01 \rangle}
    \\
  Q_{21} \sqrt{k_{1} + 1} \varphi_{k + \langle 10 \rangle}
   +
   Q_{22} \sqrt{k_{2} + 1} \varphi_{k + \langle 01 \rangle}
  \end{bmatrix}
  =
  \\
  &
  =\begin{bmatrix}
    \sqrt{\frac{2}{\epsilon}}(x_1 - q_1)\varphi_k
    -
    \overline{Q}_{11} \sqrt{k_{1}} \varphi_{k - \langle 10 \rangle}
     -
     \overline{Q}_{12} \sqrt{k_{2}} \varphi_{k - \langle 01 \rangle}
    \\
    \sqrt{\frac{2}{\epsilon}}(x_2 - q_2)\varphi_k
    -
    \overline{Q}_{21} \sqrt{k_{1}} \varphi_{k - \langle 10 \rangle}
     -
     \overline{Q}_{22} \sqrt{k_{2}} \varphi_{k - \langle 01 \rangle}
  \end{bmatrix}
  \end{split}
\end{align}
I can now multiply by $\textcolor{blue}{W\overline{\varphi_l}}$ on each side where $l \in \mathbb{N}^2$
to yield
\begin{align}
  \begin{split}
  &\begin{bmatrix}
  Q_{11} \sqrt{k_{1} + 1} \varphi_{k + \langle 10 \rangle}
   \textcolor{blue}{W\overline{\varphi_l}}
   +
   Q_{12} \sqrt{k_{2} + 1} \varphi_{k + \langle 01 \rangle}
   \textcolor{blue}{W\overline{\varphi_l}}
   \\
  Q_{21} \sqrt{k_{1} + 1} \varphi_{k + \langle 10 \rangle}
  \textcolor{blue}{W\overline{\varphi_l}}
  +
   Q_{22} \sqrt{k_{2} + 1} \varphi_{k + \langle 01 \rangle}
 \textcolor{blue}{W\overline{\varphi_l}}
 \end{bmatrix}
  =
  \\
  &
  =\begin{bmatrix}
    \sqrt{\frac{2}{\epsilon}}(x_1 - q_1)\varphi_k
    \textcolor{blue}{W\overline{\varphi_l}}
    -
    \overline{Q}_{11} \sqrt{k_{1}} \varphi_{k - \langle 10 \rangle}
    \textcolor{blue}{W\overline{\varphi_l}}
   -
     \overline{Q}_{12} \sqrt{k_{2}} \varphi_{k - \langle 01 \rangle}
    \textcolor{blue}{W\overline{\varphi_l}}
  \\
    \sqrt{\frac{2}{\epsilon}}(x_2 - q_2)\varphi_k
    \textcolor{blue}{W\overline{\varphi_l}}
    -
    \overline{Q}_{21} \sqrt{k_{1}} \varphi_{k - \langle 10 \rangle}
   \textcolor{blue}{W\overline{\varphi_l}}
   -
     \overline{Q}_{22} \sqrt{k_{2}} \varphi_{k - \langle 01 \rangle}
  \textcolor{blue}{W\overline{\varphi_l}}
  \end{bmatrix}
  \end{split}
\end{align}
Integrating on both sides gives
\begin{align}
  \begin{split}
  &\begin{bmatrix}
    Q_{11} \sqrt{k_{1} + 1} \langle \varphi_{k + \langle 10 \rangle},
    \textcolor{blue}{W\varphi_l} \rangle
   +
   Q_{12} \sqrt{k_{2} + 1} \langle \varphi_{k + \langle 01 \rangle},
   \textcolor{blue}{W\varphi_l} \rangle
   \\
  Q_{21} \sqrt{k_{1} + 1} \langle \varphi_{k + \langle 10 \rangle},
  \textcolor{blue}{W\varphi_l} \rangle
  +
   Q_{22} \sqrt{k_{2} + 1} \langle \varphi_{k + \langle 01 \rangle},
 \textcolor{blue}{W\varphi_l} \rangle
 \end{bmatrix}
  =
  \\
  &
  =\begin{bmatrix}
    \sqrt{\frac{2}{\epsilon}}\langle(x_1 - q_1)\varphi_k,
    \textcolor{blue}{W\varphi_l}\rangle
    -
    \overline{Q}_{11} \sqrt{k_{1}}\langle \varphi_{k - \langle 10 \rangle},
    \textcolor{blue}{W\varphi_l}\rangle
   -
     \overline{Q}_{12} \sqrt{k_{2}}\langle \varphi_{k - \langle 01 \rangle},
    \textcolor{blue}{W\varphi_l}\rangle
  \\
    \sqrt{\frac{2}{\epsilon}}\langle(x_2 - q_2)\varphi_k,
    \textcolor{blue}{W\varphi_l}\rangle
    -
    \overline{Q}_{21} \sqrt{k_{1}} \langle\varphi_{k - \langle 10 \rangle},
   \textcolor{blue}{W\varphi_l}\rangle
   -
     \overline{Q}_{22} \sqrt{k_{2}} \langle \varphi_{k - \langle 01 \rangle},
  \textcolor{blue}{W\varphi_l}\rangle
  \end{bmatrix}
  \end{split}
\end{align}
which gives the following recurrence relation as expected
\begin{align}
  \begin{split}
    \begin{bmatrix}
    Q_{11} & Q_{12}
    \\
    Q_{21} & Q_{22}
  \end{bmatrix}
  &\begin{bmatrix}
  \sqrt{k_{1} + 1}
  \langle \varphi_{k + \langle 10 \rangle},
    \textcolor{blue}{W\varphi_l}\rangle
    \\
  \sqrt{k_{2} + 1}
  \langle \varphi_{k + \langle 01 \rangle},
    \textcolor{blue}{W\varphi_l} \rangle
  \end{bmatrix}
  =
  \sqrt{\frac{2}{\epsilon}}
  \begin{bmatrix}
    \langle (x_1 - q_1)\varphi_k, \textcolor{blue}{W \varphi_l} \rangle
    \\
    \langle (x_2 - q_2)\varphi_k, \textcolor{blue}{W \varphi_l} \rangle
  \end{bmatrix}
  +
  \\
   &-
  \begin{bmatrix}
    \overline{Q}_{11} & \overline{Q}_{12}
    \\
    \overline{Q}_{21} & \overline{Q}_{22}
  \end{bmatrix}
  \begin{bmatrix}
    \sqrt{k_1}
    \langle\varphi_{k - \langle 10 \rangle},
    \textcolor{blue}{W\varphi_l}\rangle
    \\
    \sqrt{k_2}
 \langle\varphi_{k - \langle 01 \rangle},
   \textcolor{blue}{W\varphi_l}\rangle  \end{bmatrix}
  \end{split}
\end{align}
which we can write more coincisely as
\begin{equation}
  \begin{split}
    \bm{Q} \left( \sqrt{k_j + 1}
    \langle \varphi_{k + \langle j \rangle}, 
    W \varphi_l 
    \rangle
  \right)_{j = 1}^d
  &= 
  \sqrt{\frac{2}{\epsilon}}
  \Big( \langle (x - q)_j \varphi_k,
  W \varphi_l \rangle 
  \Big)_{j=1}^d
  +
  \\
  &
  - \bm{\overline{Q}} 
  \Big( 
    \sqrt{k_j} \langle \varphi_{k - \langle j \rangle}, W \varphi_l  \rangle
  \Big)_{j=1}^d
  \end{split}
\end{equation}
k
Assume $F_{00} = \langle \varphi_0, W \varphi_0 \rangle$ is given.
and fix $j=0$.
\begin{equation}
  \begin{split}
    F_{1j} &= \alpha \langle (x - q) \varphi_0, W \varphi_j  \rangle
    \hspace{1cm} F_{10}, F_{11}?
    \\
    F_{2j} & =
  \end{split}
\end{equation}
\\
\textcolor{blue}{Everythin should agree once you invert $Q$ - to check}
\\
Have a lattice showing ...
% CLARIFICATORY EXAMPLE FOR RECURSIVE RELATION
% TO SOLVE INTEGRALS
%
%%%%%%%%%%%%%%%%%%%%%%%%%%%%%%%%%%%
\subsection{Transformation to a tensor product form}
If the covariance matrix $PQ^{-1}$ is diagonal, then $R,I$ would be as 
well and consequently we would have a chain of $d=1$ integral, meaning for 
example that there would be no dependence on $\xi_{i}$ for $i \in \{2,...,d \}$
in the integral in $\xi_1$. In particular, this would mean that the projection 
would truly require only a one dimensional integral. 
\\
The transformation that is discussed in \cite{hagedornSymmetricKroneckerProducts2017}
transforms the Hagedorn wavepackets into a representation where the polynomials are given 
as a product of $1-d$ scaled Hermite polynomials but the complex Gaussian part remains unchanged 
with the same set of parameters and so not really suited for our case. However, this may come 
in handy if we are not able to find an expression for the Hagedorn polynomials in $d>1$...?
and numerical 
\\
If we were to transform to a Hagedorn basis with a diagonal covariance matrix (it would 
still need to satisfy lemma 1.1), then the following additional issues may arise:
%In essence, a Hagedorn wavepacket $\varphi^{\epsilon}_k$ can be re-written as 
%\begin{align}
%  \varphi_k^{\epsilon}[\Pi] = \varphi
%\end{align}
%%%%%%%%

    \section{Quantum Molecular Dynamics}
    For the purposes of the algorithms outlined in 
    this chapter we modify the assumptions for 
    the potential in the TDSE 
    \cite{schrodingerUndulatoryTheoryMechanics1926b}
    \begin{align}
          i\epsilon \partial_t \psi = H \psi \hspace{1cm}
          H = -\frac{\epsilon^2}{2}\Delta + V + W
          \label{Schrodinger}
    \end{align}
    such that  
    \begin{itemize}
      \item $V(x)$ is quadratic and (...)
      \item $W(x)$ is the non-quadratic remainder term
    \end{itemize}
    \textcolor{red}{\textbf{Questions:}}
    \begin{itemize}
      \item What are sufficient conditions to existence and uniqueness?
      \item Counterexample to existence/uniqueness?(!!)
      \item Why do we need this subquadratic growth assumption on the 
        potential?
      \item What about an example with discontinuous initial data?
        What would be the weak solution...?
    \end{itemize}

