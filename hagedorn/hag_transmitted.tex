\document[./main_hagedorn.tex]{subfiles}
\newtheorem{theorem}{Theorem}[section]
\newtheorem{corollary}{Corollary}[theorem]
\newtheorem{lemma}[theorem]{Lemma}
\usepackage{amsmath}

\begin{document}
%%%%%%%%
%
%
%       PROJECTING TRANSMITTED WAVEPACKET ONTO A HAGEDORN BASIS SET
%                 COMPUTE NEW PARAMETER SET p,Q,P 
%
%
%%%%%%%%
\subsection{Transmitted wavepacket}
We can now apply the transmission formula of equation ... for the two cases 
of constant and non-constant eigenvalues
%%%%%%%%%%%%%%%%
%
%
%   CONSTANT EIGENVALUES
%
%
%
%%%%%%%%%%%%%%%%
\subsubsection{Constant eigenvalues}
The transmitted wavepacket 
is given by \cite{betzSuperadiabaticTransitionsQuantum2009}
\begin{equation}
  \begin{split}
    \hat{\psi}^{\mp}(\xi,t_c) 
    &= 
    \exp{\left(-\frac{i}{\epsilon}S^{\pm}(t_c)\right)}
    \sin \left( \frac{\pi \gamma}{2}  \right) \times
    \\
    &
    \text{sgn}(\xi_1)
    \Theta(\xi_1^2 - 4\delta) \left(1 + \frac{\xi_1}{\nu(\xi_1)}\right)
    \exp{\left(-\frac{q_c}{\epsilon}|\xi_1 - \nu(\xi_1)|} \right)
    \times
    \\
    &
    \sum_{k \in \mathcal{K}} c_k^{\pm}(t_c) 
    \varphi_k^\epsilon[\hat{\Pi}^{\pm}(t_c)](\nu(\xi_1), \xi_2, ..., \xi_d)
    \\
    &= 
    \exp{\left(-\frac{i}{\epsilon}S^{\pm}(t_c)\right)}
    \sin \left( \frac{\pi \gamma}{2}  \right) \times
    \\
    &
    \text{sgn}(\xi_1)
    \Theta(\xi_1^2 - 4\delta) \left(1 + \frac{\xi_1}{\nu(\xi_1)}\right)
    \exp{\left(-\frac{q_c}{\epsilon}|\xi_1 - \nu(\xi_1)|} \right)
    \times
    \\
    &
    \varphi_0^\epsilon[\hat{\Pi}^{\pm}(t_c)](\nu(\xi_1), \xi_2, ..., \xi_d)
    \sum_{k \in \mathcal{K}} c_k^{\pm}(t_c)p_k(\nu(\xi_1), \xi_2, ..., \xi_d)
  \end{split}
\end{equation}
\begin{itemize}
  \item At least for Gaussian wavepackets with "large enough" momentum, the effect of the 
    cut-off function should be "negligible" as a result of the exponential decay. 
    Is this still the case as the order of the Hagedorn wavepacket increases...? The 
    variance of $\varphi^{\epsilon}_k$ scales with $k$...
\end{itemize}
%%%%%%%%
%
%   General formula - what has happened to sin(\pi \gamma / 2)
%
%%%%%%%%
\subsubsection{Non-constant + tilted crossings}
The transmitted wavepacket 
is given by \cite{betzNonadiabaticTransitionsMultiple2019} 
\begin{equation}
  \begin{split}
    \hat{\psi}^{\mp}(\xi,t_c) 
    &= 
    \exp{\left(-\frac{i}{\epsilon}S^{\pm}(t_c)\right)}
    \times
    \\
    &
    \Theta(\xi_1^2 - 4\delta) \frac{\nu(\xi_1) + \xi_1}{2|\nu(\xi_1)|}
    \exp{\left(-\frac{\tau_c}{2\delta\epsilon}|\xi_1 - \nu(\xi_1)|} \right)
    \exp{\left(-\frac{i\tau_r}{2\delta\epsilon}(\xi_1 - \nu(\xi_1))} \right)
    \times
    \\
    &
    \sum_{k \in \mathcal{K}} c_k^{\pm}(t_c) 
    \varphi_k^\epsilon[\hat{\Pi}^{\pm}(t_c)](\nu(\xi_1), \xi_2, ..., \xi_d)
  \end{split}
\end{equation}
\textcolor{red}{Is there an actual mismatch between the formulas for the constant eigenvalue 
and the general case? That is, why does the $\sin$ prefactor disappear for 
the general case? Also, I would think the formula is invariant to the direction 
of the jump between levels..? The derivation was done from up to down but intuitively 
thing should not change...?} 

\end{document}
