\document[./main_hagedorn.tex]{subfiles}
\usepackage{amsmath}
\newtheorem{theorem}{Theorem}[section]
\newtheorem{corollary}{Corollary}[theorem]
\newtheorem{lemma}[theorem]{Lemma}

\begin{document}
  \section{Non-adiabatic transitions: avoided crossings}
  \textcolor{blue}{
\textbf{Need to discuss \cite{gradinaruTunnelingDynamicsSpawning2010}}
 which is about applying a similar method but in the context of tunneling
Still need to look at the work of Hagedorn for 
gap size shrinking with $\sqrt{\epsilon}$ but also the 
the work by Olivier using normal local forms in the context 
of avoided crossings.} 
Hagedorn wavepacket dynamics in the context of avoided 
crossings has been previously investigated by Bourquin et al. in
\cite{bourquinNonadiabaticTransitionsAvoided2012} 
The authors extend the one level algorithm 
outlined in \cite{lubichQuantumClassicalMolecular2008} to the 
multilevel case: the potential matrix $V(\bm{x})$ is splitted
into a diagonal quadratic term and the non-quadratic remainder,
with the non-adiabatic coupling terms occupying the off-diagonal entries.
The same dirac-Frenkel variational principle reported in Section ...
can be applied to yield a similar set of equations for the update of the 
coefficients as in equation (...) but now it also envolves the coupling terms.
Details of the numerics regarding the computation of these integrals can be 
found in \cite{bourquinNumericalAlgorithmsSemiclassical2017}.
Here we consider incorporating the Superadiabatic formulae 
outlined in section ... into the (one level) Hagedorn dynamics framework.
A similar approach was also considered in 
\cite{bourquinNonadiabaticTransitionsAvoided2012} by implementing the 
one dimensional transition formula derived in
\cite{hagedornDeterminationNonadiabaticScattering2005} through JWKB analysis 
but with no concrete implementation given the impractility of 
the formula 
(analytic continuation - \textcolor{red}{needs to be explained better}).
\textcolor{red}{eventually will need a comparison of the numerical results
beside a description of advantages/disadvantages}
Since the transmitted wavepacket is expressed in momentum space,
We will now denote the parameter set as $\hat{\Pi} := \{p, q, P, Q\}$
as a reminder.
\\
\textbf{Initial condition}
\\
Consider the following initial condition corresponding to one of the adiabatic
subspaces
\begin{align}
  \hat{\psi}^{\pm}(\xi,0) := 
  \sum_{k \in \mathcal{K}} c_k^{\pm}(0) \varphi_k^\epsilon[\hat{\Pi}^{\pm}(0)](\xi)
\end{align}
where $c_k^{\pm} \in \mathbb{C}$ \xi \in \mathbb{R}^d$
and the index set $\mathcal{K} \subset \mathbb{N}^d$.
Let $t_c$ denote the time at which the avoided crossing is detected. 
The wavepacket at time $t_c$
is given by 
\begin{align}
  \hat{\psi}^{\pm}(\xi,t_c) = 
  \exp{\left(-\frac{i}{\epsilon}S^{\pm}(t_c)\right)}
  \sum_{k \in \mathcal{K}} c_k^{\pm}(t_c) \varphi_k^\epsilon[\hat{\Pi}^{\pm}(t_c)](\xi)
\end{align}
The projection onto a Hagedorn basis set at the avoided crossing for the 
transmitted wavepacket requires the following steps:
\begin{enumerate}
  \item \textbf{Change of basis/co-ordinates} for the Hagedorn wavepackets such that
    a coordinate axis is parallel to $p(t_c)$, the mean momentum of the wavepacket 
    at the time of the avoided crossing (and keep same orientation)
  %\item \textbf{Transform a wavepacket into a tensor product form} (see Hagedorn
  %  and Lasser paper) - assuming we can get a representation for the 
  %  $d>1$ polynomials then we will not need this if attempting an 
  %  analytical approximation to the method
  \item \textbf{Compute the new parameter set} for $p,P,Q$ 
  likely change $p$ but keep the same $P$ and $Q$ 
  \item \textbf{Project the transmitted wavepackets onto the new basis set.}
\end{enumerate}
\end{document}

