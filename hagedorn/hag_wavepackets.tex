\document[./main_hagedorn.tex]{subfiles}
\usepackage{amsmath}
\usepackage{amssymb}
\newtheorem{theorem}{Theorem}[section]
\newtheorem{lemma}[theorem]{Lemma}
\theoremstyle{definition}
\let\proof\relax\let\endproof\relax % use proof from amsthm
\usepackage{amsthm,xpatch}
\xpatchcmd{\proof}{\itshape}{\bfseries}{}{}
\renewcommand\qedsymbol{$\blacksquare$}
\usepackage{comment}
\usepackage[
backend=biber,
style=alphabetic,
sorting=ynt
]{biblatex}
\addbibresource{Hagedorn.bib}

\begin{document}

\subsection{Hagedorn's parametrization}
\textcolor{blue}{To do list:
\begin{itemize}
  \item Why is it exact for a quadratic potential (metaplectic transformation?) - (similar to egorov theorem)  
  \item what does $e^{-iH/t}$ do qualitatively? (exponentiation of 1st order derivative is translation)
  \item How does the variance of the wavepacket evolve in time? Can you draw 
    any qualitative conclusions about when the support spreads out fast vs slow?
%  \item It would be interesting to understand more about this symplecticity 
%    notion
%  \item Make sure you understand the evolution of the parameters qualitatively
%  \item Exercise: proofs - what is the error, it should go down to zero if I consider 
%    an infinite number of wavepackets...? it should converge
%  \item the error in the numerical scheme... 
%  \item this classical action phase... why is it there...?
%  \item Construction of these Hagedorn wavepackets from raising and lowering operators?
%    what is the motivation for introducing raising and lowering operators...?
%  \item Quantitative/qualitative description of Hagedorn wavepackets - variance increases 
%    with $k$... what else...
%  \item Symplectic relation
\end{itemize}
}
Consider a complex semi-classical Gaussian wavepacket in the following form
  \begin{align}
    \varphi^\epsilon_0(\bm{x}):=
    (\pi\epsilon)^{-\frac{d}{4}} (\det
    \bm{Q})^{-\frac{1}{2}}
    \exp{\left( \frac{i}{2\epsilon}(\bm{x}-\bm{q})^T\bm{PQ}^{-1}(\bm{x}-\bm{q}) +
    \frac{i}{\epsilon}\bm{p}^T(\bm{x}-\bm{q})\right)}
    \label{hagedorn:hagedorn_wavepackets:complex_gaussian}
  \end{align}
  Here on we will follow the notational convention
as in \cite{bourquinNumericalAlgorithmsSemiclassical2017} to denote the set 
of parameters as $\Pi := \left(\bm{q}, \bm{p}, \bm{Q}, \bm{P}\right)$. 
The matrix $\mathbb{C}^{d \times d} \ni \bm{C} := \bm{PQ}^{-1}$ is complex symmetric 
with positive definite imaginary part. The matrices $\bm{P}$ and $\bm{Q}$ must 
satisfy further properties as enunciated in the following Lemma. 
\begin{lemma}[\cite{hagedornRaisingLoweringOperators1998}]
  \label{hagedorn:hagedorn_orthogonal_projection:lemma:symplectic_equations} 
  Two square matrices $\bm{Q}, \bm{P} \in \mathbb{C}^{d \times d}$ satisfy the following relations 
  \begin{equation} 
    \begin{split}
      \bm{Q}^T\bm{P} - \bm{P}^T\bm{Q}   & = 0 \\
      \bm{Q}^*\bm{P} - \bm{P}^*\bm{Q} & = 2i\bm{I}_{d}
    \end{split}
    \label{hagedorn:hagedorn_wavepackets:lemma:symplectic:equations}
  \end{equation}
  if and only if $\bm{Q}$ and $\bm{P}$ are invertible and $\bm{C}=\bm{PQ}^{-1}$ is complex symmetric with positive 
  definite imaginary part. Moreover, $\Im(\bm{C}) = (\bm{QQ}^*)^{-1}$.
\end{lemma}
\begin{proof}
  $\left( \Rightarrow  \right)$ Suppose b.w.o.c. that $Q$ is 
  not invertible. Then $\det(\bm{Q}) = 0 = \det(\bm{Q}^*)$
  which leads to 
  a contradiction when taking the determinant of both sides of the 
  second equation in \eqref{hagedorn:hagedorn_wavepackets:lemma:symplectic:equations}.
  Similarly for $\bm{P}$.
  \\
  Multiplying the first equation in 
  \eqref{hagedorn:hagedorn_wavepackets:lemma:symplectic:equations}
  by $\bm{Q}^{-1}$ on the right we get $\bm{Q}^T\bm{PQ}^{-1} = \bm{P}^T \Rightarrow 
  \bm{PQ}^{-1} = (\bm{Q}^T)^{-1}\bm{P}^T = (\bm{PQ}^{-1})^T$ and therefore $\bm{C}$ is complex 
  symmetric.
\end{proof}
Show positive definiteness: $\bm{x}^T \Im(\bm{C}) \bm{x} = \langle \bm{x}, \Im(\bm{C})\bm{x}  \rangle > 0
\text{ } \forall \text{ } \bm{x} \in \mathbb{R}^n$
\\
$\left( \Leftarrow \right)$
\\\\
Lemma ... gives a sufficient and necessary condition for two matrices
$\bm{Q},\bm{P}$ to yield such a matrix $\bm{C}$ (?) and also a method(?) to construct 
two desired such matrices.
\textcolor{red}{Given a complex symmetric matrix, how to decompose into such 
P, Q. Is there a trivial decomposition involving say the identity matrix. However, 
is there a decomposition which requires a least number of wavepackets such that 
the error in the norm (or another statistic/observable) is less than some tolerance value?}
\\
\\
The fact that $\Im(\bm{C}) = (\bm{QQ}^{*})^{-1}$ justifies why the normalisation factor 
of the complex Gaussian depends on $\bm{Q}$ alone.
The reason why Hagedorn's parametrisation is useful will become clearer when 
we will look at the dynamics. It is then useful to compare it to what one obtains when 
this parametrisation is not made. 
\end{document}
