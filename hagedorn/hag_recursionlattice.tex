\document[./main_hagedorn.tex]{subfiles}
\usepackage{tikz}
\usetikzlibrary{calc}


\begin{document}
\begin{figure}[h!]
  \centering
  \begin{tikzpicture}
    \coordinate (Origin)   at (0,0);
    \coordinate (Origin1)   at (-0.5,0);
    \coordinate (XAxisMin) at (0,0);
    \coordinate (XAxisMax) at (6,0);
    \coordinate (YAxisMin) at (0,0);
    \coordinate (YAxisMax) at (0,5);
    \draw [thin, gray,-latex] (XAxisMin) -- (XAxisMax);% Draw x axis
    \draw [thin, gray,-latex] (YAxisMin) -- (YAxisMax);% Draw y axis

    \clip (-1.5,-0.5) rectangle (6cm,6cm); % Clips the picture...
    \pgftransformcm{1}{0.0}{0.0}{0.5}{\pgfpoint{0cm}{0cm}}
          % This is actually the transformation matrix entries that
          % gives the slanted unit vectors. You might check it on
           % MATLAB etc. . I got it by guessing.
    \coordinate (Bone) at (0,3.5);
    \coordinate (Btwo) at (3.5,0);
    \coordinate (K11) at (3.5,3.5);
    \coordinate (Bthree) at (4,4);
    \coordinate (X) at (5.5,-1.3);
    \coordinate (Y) at (-0.5,6.8);
    \draw[style=help lines,dashed] (-14,-14) grid[step=3.5cm] (14,14);
          % Draws a grid in the new coordinates.
          %\filldraw[fill=gray, fill opacity=0.3, draw=black] (0,0) rectangle (2,2);
              % Puts the shaded rectangle
    \foreach \x in {0,1,...,4}{% Two indices running over each
      \foreach \y in {-1,0,...,4}{% node on the grid we have drawn 
        \node[draw,circle,inner sep=2pt,fill] at (3.5*\x,3.5*\y) {};
            % Places a dot at those points
      }
    }
    \draw [ultra thick,-latex,blue] (Origin)
    -- (Bone) node [above left]; % {$\varphi_{01}$};
    \draw [ultra thick,-latex, blue] (Origin)
    -- (Btwo) node [below right]; % {$\varphi_{10}$};
    %\draw [ultra thick,-latex, red] (Bone)
    %-- (K11) node [below right]; % {$\varphi_{10}$};
    %\draw [ultra thick,-latex, green] (Btwo)
    %-- (K11) node [below right]; % {$\varphi_{10}$};
    %\node[draw,circle,inner sep=2pt,fill, blue] at (Bthree) {};
    %\coordinate [label=$q_kp_k$] (A) at (Bthree) ; 
  \coordinate [label=\Large$\bm{\varphi^\epsilon_{k_1, k_2}}$] (A) at (Origin1) ; 
  \coordinate [label=\Large\textcolor{blue}{$\bm{\varphi^\epsilon_{k_1, k_2 + 1}}$}] (B) at (Bone) ; 
  \coordinate [label=\Large$\textcolor{blue}{\bm{\varphi^\epsilon_{k_1 + 1, k_2}}}$] (B) at (Btwo) ; 
  \coordinate [label=\Large$k_1$] (C) at (X) ; 
  \coordinate [label=\Large$k_2$] (D) at (Y) ; 
  % \draw [ultra thick,-latex,red] (Origin)
   %     -- ($(Bone)+(Btwo)$) node [below right] {$b_1+b_2$};
   % \draw [ultra thick,-latex,red] (Origin)
   %     -- ($2*(Bone)+(Btwo)$) node [above left] {2$b_1+b_2$};
   % \filldraw[fill=gray, fill opacity=0.3, draw=black] (Origin)
   %     rectangle ($2*(Bone)+(Btwo)$);
    %\draw [thin,-latex,red, fill=gray, fill opacity=0.3] (0,0)
        % -- ($2*(0,2)+(2,-2)$)
        % -- ($3*(0,2)+2*(2,-2)$) -- ($(0,2)+(2,-2)$) -- cycle;
  \end{tikzpicture}
  \caption{Graphical representation of the recurrence relation in $d=2$. For 
  wavepacket with index $(i,j)$, there are how many paths...? to construct it using 
  the recurrence relation.}
  \label{figure:lattice}
\end{figure}
\end{document}

