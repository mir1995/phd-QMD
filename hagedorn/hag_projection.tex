\document[./main_hagedorn.tex]{subfiles}
\renewcommand{\baselinestretch}{1.5}
\usepackage{comment}
\usepackage[
backend=biber,
style=alphabetic,
sorting=ynt
]{biblatex}
\addbibresource{Hagedorn.bib}

\begin{document}

%%%%%%%%
%
%
%       PROJECTING TRANSMITTED WAVEPACKET ONTO A HAGEDORN BASIS SET
%
%
%%%%%%%%
%%%%%%%%
%
%
%       PROJECTING TRANSMITTED WAVEPACKET ONTO A HAGEDORN BASIS SET
%                 CHANGE OF BASIS/CARTESIAN CO-ORDINATES
%
%
%%%%%%%%
%%%%%%%%
%
%
%       PROJECTING TRANSMITTED WAVEPACKET ONTO A HAGEDORN BASIS SET
%                 PROJECTION BY COMPUTING INNER PRODUCT
%
%
%%%%%%%%
\subsection{Projecting onto Hagedorn basis}
Upon detection of the avoided crossing and application of the transmission 
formula, the transmitted wavepacket $\hat{\psi}^{\mp}$ needs to be projected 
back onto a Hagedorn basis set $\Phi[\hat{\Pi}^\prime] =
\{\varphi_l[\hat{\Pi}^\prime]\}_{l \in \mathcal{L}}$ 
in order for it to be evolved away from the crossing using Hagedorn dynamics.  
Since $\Phi[\hat{\Pi}^\prime]$ is an orthonormal set,
the closest point in its linear span to the transmitted 
wavepacket is given by 
\begin{equation}
  P_{\Phi}\hat{\psi}^\mp := 
  \sum_{l \in \mathcal{L}} \left \langle \frac{\hat{\psi}^\mp}{\|\hat{\psi}^\mp \|} ,
  \varphi_l[\hat{\Pi}^\prime]  
  \right \rangle \varphi_l[\hat{\Pi}^\prime]
\end{equation}
and it is unique since a linear subspace is convex.
The coefficients 
$a_l := \left \langle \frac{\hat{\psi}^\mp}{\|\hat{\psi}^\mp \|} ,
\varphi_l[\hat{\Pi}^\prime] \right \rangle $
are known as Fourier coefficients.
%\begin{equation} 
%  \begin{split}
%    \\
%        =&\int_{\mathbb{R}^d} \psi \text{ } 
%    \overline{\varphi_l[\hat{\Pi}^\prime]} d\xi
%  \end{split}
%\end{equation}
The parameters $\hat{\Pi}$ for the new sub-basis set may take different 
values than the ones for the incoming wavepacket so as to minimise 
the number of Fourier coefficients $a_l$ needed to represent the 
transmitted wavepacket. This is further discussed in the next 
subsection.\textcolor{red}{explain better} 
%%%%%%%%
%
%
%       PROJECTING TRANSMITTED WAVEPACKET ONTO A HAGEDORN BASIS SET
%                 COMPUTE NEW PARAMETER SET p,Q,P 
%
%
%%%%%%%%

\subsubsection{Different parameter set $\Pi$}
\textbf{Constant eigenvalues - do it for the general case }
\\
Given the considerations outlined in the previous subsection 
we consider a different parameter set for the Hagedorn basis 
corresponding to the transmitted wavepacket. More precisey we change 
the entry of the momentum parameter $p$ corresponding to the direction of motion 
of the wavepacket at the crossing, 
$\{p, q, P, Q \} \mapsto \{ p^\prime, q, P, Q \} $ where 
$p^\prime_{i} = p_{i} \text{ } \forall \text{ } 1 < i  \leq d$
while
\begin{equation}
  \begin{split}
    p_1^\prime &= \langle \xi \hat{\psi}^\mp(\xi_1), \hat{\psi}^{\mp}(\xi_1) \rangle    
    \\
    &= ...
  \end{split}
\end{equation}
 The Fourier coefficients are then given by 
\begin{equation}
  \begin{split}
    a_l =& \left \langle \frac{\hat{\psi}^\mp}{\|\hat{\psi}^\mp \|} ,
      \varphi_l[\hat{\Pi}^\prime] \right \rangle 
    \\
    &=
    \frac{1}{\|\hat{\psi}^\mp \|}
    \left \langle \sum_{k \in \mathcal{K}}c^{\pm}_{k}(t_c) 
    f(\xi_1, \nu(\xi_1); \delta)
    \varphi_k^\epsilon[\hat{\Pi}^{\pm}_{t_c}](\tilde{\xi}), 
  \varphi_l^\epsilon[\hat{\Pi}^\prime](\xi)  \right \rangle
    \\
    &=
    \frac{1}{\|\hat{\psi}^\mp \|}
    \sum_{k \in \mathcal{K}}c^{\pm}_{k}(t_c)
    \left \langle  
    f(\xi_1, \nu(\xi_1); \delta)
    \varphi_k^\epsilon[\hat{\Pi}^{\pm}_{t_c}](\tilde{\xi}), 
  \varphi_l^\epsilon[\hat{\Pi}^\prime](\xi)  \right \rangle
    %\\
    %&=
    %\frac{1}{\|\hat{\psi}^\mp \|}
    %\sum_{k \in \mathcal{K}}c^{\pm}_{k}(t_c)
    %\left \langle  
    %f(\xi_1, \nu(\xi_1); \delta)
    %P_k[\hat{\Pi}^{\pm}_{t_c}](\tilde{\xi})
    %\varphi_0^\epsilon[\hat{\Pi}^{\pm}_{t_c}](\tilde{\xi}), 
    %P_l[\hat{\Pi}^{\pm}_{t_c}](\xi) 
    %\varphi_0^\epsilon[\hat{\Pi}^\prime](\xi)  \right \rangle
  \end{split}
\end{equation}
where 
\[
  f(\xi_1, \nu(\xi_1); \delta) 
  =
    \exp{\left(-\frac{i}{\epsilon}S^{\pm}(t_c)\right)}
    \sin \left( \frac{\pi \gamma}{2}  \right) 
    \text{sgn}(\xi_1)
    \Theta(\xi_1^2 - 4\delta) \left(1 + \frac{\xi_1}{\nu(\xi_1)}\right)
    \exp \left[- \frac{q_c}{\epsilon}|\xi_1 - \nu(\xi_1)| \right]
\]
and 
\[
  \tilde{\xi} = (\nu(\xi_1), \xi_2, ..., \xi_d) 
\]
It is actually convenient to define 
\[
  a_{kl} :=  
    \left \langle  
    f(\xi_1, \nu(\xi_1); \delta)
    %P_k[\hat{\Pi}^{\pm}_{t_c}](\tilde{\xi})
    \varphi_k^\epsilon[\hat{\Pi}^{\pm}_{t_c}](\tilde{\xi}), 
    %P_l[\hat{\Pi}^{\pm}_{t_c}](\xi) 
    \varphi_l^\epsilon[\hat{\Pi}^\prime](\xi)  \right \rangle
\]
so that 
\[
  a_l =  
    \frac{1}{\|\hat{\psi}^\mp \|}
    \sum_{k \in \mathcal{K}}c^{\pm}_{k}(t_c)
    a_{kl}
\]
Following the same approach as for the update of the coefficients 
stemming from the variational approximation, computation of 
the coefficients requires solving integrals of the following form 
\begin{align}
  \begin{split}
    kd
  \end{split}
\end{align}
\newpage
%Our objective (not only) is to study the behaviour of the Fourier coefficients 
%as a function of $l \in \mathcal{L}$ for fixed $k \in \mathcal{K}$.
%\\
%\\
Under the integral sign, the product of the two Gaussians is given by 
\begin{equation}
  \begin{split}
    &= 
    \exp{\left[ 
        \frac{i}{2\epsilon}(\tilde{\xi}-p)^TQP^{-1}(\tilde{\xi}-p) +
      \frac{i}{\epsilon}q^T(\tilde{\xi}-p)
  -\frac{q_c}{\epsilon}|\xi_1 - \nu(\xi_1)| \right] }
  \\
    &\hspace{1cm}
    \exp{ \left[
      \overline{\frac{i}{2\epsilon}(\xi-p^\prime)^TQP^{-1}(\xi-p^\prime) +
      \frac{i}{\epsilon}q^T(\xi-p^\prime)}
    \right] } \text{ } d \xi
  \end{split}
\end{equation}
We can simplify the argument in the exponential term further. Consider the 
real and imaginary matrix decomposition $QP^{-1} = R + i I$ where $R$ and $I$
are symmetric since $QP^{-1}$ is, and let
\begin{equation}
  \tilde{a} = [\nu(\xi_1) - p_1, 0, ..., 0] 
  \hspace{.8cm}
  a = [\xi_1 - p_1^\prime, 0, ..., 0] 
  \hspace{.8cm}
  b = [0, \xi_2 - p_2, ..., \xi_d - p_d] 
\end{equation}. Then,
\begin{equation}
  \begin{split}
   &     \frac{i}{2\epsilon}(\tilde{a} + b)^T(R + i I)(\tilde{a} + b) +
   \frac{i}{\epsilon}q^T(\tilde{a} + b)
  -\frac{q_c}{\epsilon}|\xi_1 - \nu(\xi_1)| +
  \\
  &
  + \overline{\frac{i}{2\epsilon}(a + b)^T(R + iI)(a + b) +
      \frac{i}{\epsilon}q^T(a + b)} = 
  \\
  &
  -\frac{1}{2\epsilon}\left[ 
    (\tilde{a} + b)^T I(\tilde{a} + b) 
    + (a + b)^T I(a + b) 
  + 2q_c |\xi_1 - \nu(\xi_1)| 
  \right]
  \\
  &
  +\frac{i}{2\epsilon}\left[ 
    (\tilde{a} + b)^T R(\tilde{a} + b) 
    - (a + b)^T R(a + b) + 2q^T(\tilde{a} - a) 
  \right]
  =
  \\
  &
  -\frac{1}{2\epsilon}\left[ 
    \tilde{a}^T I \tilde{a} + a^T Ia 
    + 2b^T I b + (\tilde{a}^T + a^T)I b
    + b^TI(\tilde{a} + a)
  + 2q_c |\xi_1 - \nu(\xi_1)| 
  \right]
  \\
  &
  +\frac{i}{2\epsilon}\left[ 
    \tilde{a}^T R \tilde{a} - a^T Ra 
     + (\tilde{a}^T - a^T)R b
    + b^TR(\tilde{a}  - a)
    + 2q_1(\tilde{a}_1 - a_1)) 
  \right]
  =
  \\
  &
  -\frac{1}{2\epsilon}\left[ 
  \tilde{a}^T I \tilde{a} + a^T Ia 
    + 2b^T I b + 2(\tilde{a}^T + a^T)I b
  + 2q_c |\xi_1 - \nu(\xi_1)| 
  \right]
  \\
  &
  +\frac{i}{2\epsilon}\left[ 
    \tilde{a}^T R \tilde{a} - a^T Ra 
     + 2(\tilde{a}^T - a^T)R b
    + 2q_1(\tilde{a}_1 - a_1) 
  \right]
  =
  \\
  &
  -\frac{1}{2\epsilon}\left[ 
    I_{11}(\tilde{a}_1^2 + a_1^2) 
    + 2b^T I b + 2(\tilde{a}_1 + a_1)\sum_{j = 2}^d I_{1,j} b_j
  + 2q_c |\xi_1 - \nu(\xi_1)| 
  \right]
  \\
  &
  +\frac{i}{2\epsilon}\left[ 
    R_{11} (\tilde{a}_1^2 - a_1^2)  
    + 2(\tilde{a}_1 - a_1)\sum_{j=2}^d R_{1,j} b_j
    + 2q_1(\tilde{a}_1 - a_1) 
  \right]
  =
  \\
  &
  -\frac{1}{2\epsilon}\left[ 
    I_{11}(\tilde{a}_1^2 + a_1^2) 
    + 2b^T I b + 2(\tilde{a}_1 + a_1)\sum_{j = 2}^d I_{1,j} b_j
  + 2q_c |\xi_1 - \nu(\xi_1)| 
  \right]
  \\
  &
  +\frac{i}{2\epsilon}\left[ 
    R_{11} (\tilde{a}_1^2 - a_1^2)  
    + 2(\tilde{a}_1 - a_1)\left(q_1 + \sum_{j=2}^d R_{1,j} b_j\right) 
  \right]
  \end{split}
\end{equation}
We investigate different attempts to solving the integral for the Fourier coefficients. 
For each approach there is a convenient way of re-writing the integral.
\\
\\
The integral representation of $a_{kl}$ is given by  
\begin{equation}
  \begin{split}
    a_{kl} &\propto 
    \int_{\mathbb{R}^{d}} \text{sgn}(\xi_1)
      \Theta(\xi_1^2 - 4\delta)\left( 1 + \frac{\xi_1}{\nu(\xi_1)} \right)
      P_k[\hat{\Pi}^{\pm}_{t_c}](\tilde{\xi})
      \overline{ P_l[\hat{\Pi}^\prime](\xi)} 
      \times 
      \\
      &
      \exp{\left\{
    -\frac{1}{2\epsilon}\left[ 
      I_{11}(\tilde{a}_1^2 + a_1^2) 
      + 2b^T I b + 2(\tilde{a}_1 + a_1)\sum_{j = 2}^d I_{1,j} b_j
    + 2q_c |\xi_1 - \nu(\xi_1)| 
    \right]
      \right\} }
    \\
    &
    \exp{\left\{
    \frac{i}{2\epsilon}\left[ 
      R_{11} (\tilde{a}_1^2 - a_1^2)  
      + 2(\tilde{a}_1 - a_1)\left(q_1 + \sum_{j=2}^d R_{1,j} b_j\right) 
    \right]
      \right\} }  d\xi
  \\
    &=\int_{\mathbb{R}^{d-1}}  
      \exp{\left\{
    -\frac{1}{\epsilon}\left[ 
       b^T I b 
    \right]
      \right\} }
      \int_{\mathbb{R}} \text{sgn}(\xi_1)
      \Theta(\xi_1^2 - 4\delta)\left( 1 + \frac{\xi_1}{\nu(\xi_1)} \right)
      P_k[\hat{\Pi}^{\pm}_{t_c}](\tilde{\xi})
      \overline{ P_l[\hat{\Pi}^\prime](\xi)} 
      \\
      &
      \times 
      \exp{\left\{
    -\frac{1}{2\epsilon}\left[ 
      I_{11}(\tilde{a}_1^2 + a_1^2) 
       + 2(\tilde{a}_1 + a_1)\sum_{j = 2}^d I_{1,j} b_j
    + 2q_c |\xi_1 - \nu(\xi_1)| 
    \right]
      \right\} }
    \\
    &
    \exp{\left\{
    \frac{i}{2\epsilon}\left[ 
      R_{11} (\tilde{a}_1^2 - a_1^2)  
      + 2(\tilde{a}_1 - a_1)\left(q_1 + \sum_{j=2}^d R_{1,j} b_j\right) 
    \right]
\right\} }  d\xi_1d\xi_{d-1}
  \end{split}
\end{equation}
The proportionality constant is $\exp{\left( -\frac{i}{\epsilon} S^{\pm}(t_c) \right)} \sin(\frac{\pi \gamma}{2})$
\textcolor{red}{Have the resulting integral here}
\\
We also note that $I$ is real symmetric and so diagonalizable
%%%%%%%%%%%%
%
%
% TAYLOR EXPANSION + GAUSSIAN INTEGRALS
%
%
%%%%%%%%%%%%
%%%%%%%%
%
%
%       PROJECTING TRANSMITTED WAVEPACKET ONTO A HAGEDORN BASIS SET
%                 TRANSFORMATION TO A TENSOR PRODUCT FORM
%
%
%%%%%%%%
\subsection{d-dimensional case}
Solution of the inner most nasty integral in $\xi_1$ may then allow 
us to solve the remaining $d-1$ dimensional integral explicitly
\subsubsection{Monte Carlo Integration}
(Ignore global phase factor for the moment)
\\
We find it convenient to re-write $a_{kl}$ in terms of complex 
coefficients as follows (we will use this form in other parts 
of the projection section)
\begin{equation}
  \begin{split}
    a_{kl} &= \sin\left(\frac{\pi \gamma}{2}\right) 
    \int_{\mathbb{R}^{d-1}} \exp{\left[ -\frac{1}{\epsilon}b^TIb  \right]}
    \int_{\mathbb{R}} \text{sgn}(\xi_1) \Theta(\xi_1^2 - 4\delta)\left( 1 + \frac{\xi_1}{\nu(\xi_1)} \right)
      P_k[\hat{\Pi}^{\pm}_{t_c}](\tilde{\xi})
      \overline{ P_l[\hat{\Pi}^\prime](\xi)} 
    \times 
    \\
    &
    \hspace{1cm}
    \exp{\left[-\frac{1}{2\epsilon} 
        \left( \alpha \tilde{a}_1^2 + \overline{\alpha} a_1^2
        + \beta \tilde{a}_1 + \overline{\beta} a_1 + 2q_c|\xi_1 - \nu(\xi_1)|
      \right) 
    \right]}  d \xi_1 d\xi_{d-1}
    \\
    &= \int_{\mathbb{R}^{d-1}} \exp{\left[ -\frac{1}{\epsilon}b^TIb  \right]}
    \int_{\mathbb{R}} \text{sgn}(\xi_1) \Theta(\xi_1^2 - 4\delta)\left( 1 + \frac{\xi_1}{\nu(\xi_1)} \right)
      P_k[\hat{\Pi}^{\pm}_{t_c}](\tilde{\xi})
      \overline{ P_l[\hat{\Pi}^\prime](\xi)} 
    \times 
    \\
    &
    \hspace{1cm}
    \exp \Bigg[-\frac{1}{2\epsilon} 
        \Big( \alpha (\nu(\xi_1) - p_1)^2 + \overline{\alpha} (\xi_1 - p_1^\prime)^2
          + \beta (\nu(\xi_1) - p_1) + \overline{\beta} (\xi_1 - p_1^\prime)
    \\
    &
    \hspace{1cm}
    + 2q_c|\xi_1 - \nu(\xi_1)|
      \Big) 
    \Bigg]  d \xi_1 d\xi_{d-1}
    \\
  \end{split}
\end{equation}
We spot a Gaussian term in the integral which would be suitable for 
numerical integration via MC.
\begin{equation}
  \begin{split}
    a_{kl}&= \int_{\mathbb{R}^{d-1}} \exp{\left[ -\frac{1}{\epsilon}b^TIb  \right]}
    \int_{\mathbb{R}} \text{sgn}(\xi_1) \Theta(\xi_1^2 - 4\delta)\left( 1 + \frac{\xi_1}{\nu(\xi_1)} \right)
      P_k[\hat{\Pi}^{\pm}_{t_c}](\tilde{\xi})
      \overline{ P_l[\hat{\Pi}^\prime](\xi)} 
    \times 
    \\
    &
    \hspace{1cm}
    \exp \Bigg[-\frac{1}{2\epsilon} 
        \Big( \alpha (\nu(\xi_1) - p_1)^2 
          + \beta (\nu(\xi_1) - p_1) + \overline{\beta} (\xi_1 - p_1^\prime)
       + 
       iR_{11}
    (\xi_1 - p_1^\prime)^2  
        + 2q_c|\xi_1 - \nu(\xi_1)|
    \Big) \Bigg]
    \\
    &
    \hspace{1cm}
    \textcolor{blue}{
    \exp \Bigg[
      -\frac{I_{11}}{2\epsilon}
    (\xi_1 - p_1^\prime)^2  
\Bigg] } d \xi_1 d\xi_{d-1}
  \end{split}
\end{equation}
where 
\begin{equation}
  \begin{split}
  \alpha = I_{11} - i R_{11} &\hspace{1cm}  
    \beta = -i2q_1 + 2\sum_{j=2}^d (I_{1,j}  - i R_{1,j})b_j 
  \end{split}
\end{equation}
We can use a Monte Carlo estimator 
so that 
\begin{equation}
  \begin{split}
    a_{kl} =& \lim_{N\to\infty}
    \sum_{n=0}^N
    \text{sgn}(\xi_n) \Theta(\xi_n^2 - 4\delta)
    \left( 1 + \frac{\xi_n}{\nu(\xi_n)} \right)
    \\
            &
    \exp \Bigg[-\frac{1}{2\epsilon} 
        \Big( \alpha (\nu(\xi_n) - p_1)^2 
       + 
       iR_{11}
    (\xi_n - p_1^\prime)^2  
        + 2q_c|\xi_n - \nu(\xi_n)|
    \Big) \Bigg]
    \times
    \\
    &\int_{\mathbb{R}^{d-1}} 
    P_k[\hat{\Pi}^{\pm}_{t_c}](\nu(\xi_n), \xi_2, ..., \xi_d)
      \overline{ P_l[\hat{\Pi}^\prime](\xi_n, \xi_2, ..., \xi_d)} 
      \\
    &
    \exp{\left[ -\frac{1}{\epsilon}(b^TIb  )  \right]}
  \end{split}
\end{equation}
To do:
\begin{itemize}
  \item integral is incomplete but we get the idea
  \item will loose efficiency of samples 
    when checking $\Theta(....)$
  \item there is some normalisation constant 
    missing
  \item inner integral should be solvable since $I$ is 
    diagonalizable
\end{itemize}
%%%%%%%%%%%%%%%%%%%%
%
% 
%     D DIMENSIONAL INTEGRAL APPROXIMATION VIA FIRST ORDER 
%                     TAYLOR EXPANSION
%
%
%%%%%%%%%%%%%%%
\subsubsection{Taylor expansion + Gaussian Integrals}
We will consider this approach once we have an expression for the $k^\text{th}$ 
order polynomials. For the moment jump to the section on one dimensional 
integration which consists in the same approach and not much will change 
for this situation since we will be able to solve the remaining $d-1$ 
integrals exactly.
%%%%%%%%%%%%%%%%%%%%%%%%%
%
%
%     PRINCIPLE OF STATIONARY PHASE - ONE DIMENSION 
%
%
%%%%%%%%%%%%%%%%%%%%%%%%
\subsubsection{One dimension - principle of stationary phase}
\\
Could think of solvin inner integral using an asymptotic expansion 
and then see whether the resulting $d-1$ integral could be solved exactly...?
\\
In this case we want to re-write the integral in a form such that the imaginary term in 
the exponent is explicit. For $d=1$, the polynomials reduce to Hermite polynomials.
The integral is of the form  
\begin{equation}
  \begin{split}
    \int_{\mathbb{R}} f(\xi; \delta) \exp{\left[ \frac{i}{2\epsilon} g(\xi; \delta)  \right]} d \xi
  \end{split}
\end{equation}
where 
\begin{equation}
  \begin{split}
      f(\xi; \delta) &= \text{sgn}(\xi_1) \Theta(\xi_1^2 - 4\delta)\left( 1 + \frac{\xi_1}{\nu(\xi_1)} \right)
      p_k[\Pi](\nu(\xi)) p_l[\Pi^\prime](\xi)
      \\
      &
      \exp{\left[ - \frac{1}{2\epsilon} (2 I_{11} \xi_1^2 
          + 2\xi_1(-I_{11}p_1^\prime + \sum_{j=2}^d I_{1,j}b_j) 
          + 2\nu(\xi_1)(-p_1I_{11} + \sum_{j=2}^dI_{1,j}b_j)
      )  \right]}
          \\
      &
         \exp{\left[
             I_{11}(-4\delta + p_1^2 + p_1^\prime^2) 
             - 2 (p_1 + p_1^\prime)(\sum_{j=2}^dI_{1,j}b_j)
             + q_c|\xi_1 - \nu(\xi_1)|
      \right]}
      \\
      g(\xi; \delta) &= 2\nu(\xi_1)( \sum_{j=2}^dR_{1,j}b_j + q_1 - R_{11}p_1  ) 
      - 2\xi_1(\sum_{j=2}^d R_{1,j}b_j + q_1  - R_{11}p_1^\prime )
       \\
      & R_{11}(- 4\delta + p_1^2 - p_1^\prime^2)
      + 2\left( q_1 + \sum_{j = 2}^d R_{1,j}b_j \right) (- p_1 + p_1^\prime ) 
  \end{split}
\end{equation}
\textcolor{red}{If you approximate $\nu(\xi_1)$ then you would be able 
to say something about the frequency of the oscillations}
\\
For $d = 1$, the polynomials reduce to the Hermite polynomials 
We can re-write the oscillator more succintly as 
\[ 
  g(\xi_1; \delta) = \alpha \nu(\xi_1) + \beta \xi_1 + \gamma 
\]
where 
\begin{equation}
  \begin{split}
    \alpha &= 2( q_1 + \sum_{j=2}^d R_{1,j}b_j - Rp_1 )  
  \\
  \beta &= 2(- q_1 - \sum_{j=2}^d R_{1,j}b_j + Rp_1^\prime)
  \\
  \gamma &=  
\end{split}
\end{equation}
and $g(\xi)$ has a stationary point on $(2\sqrt{\delta}, \infty)$ at $\xi_1^*$
that solves
\begin{equation}
  \begin{split}
    &\frac{\alpha \xi_1}{\sqrt{\xi_1^2 - 4\delta}}
    + \beta = 0 
    \\
    &
    \Leftrightarrow
    \xi_1^* = \sqrt{\frac{-4\beta^2 \delta}{\alpha^2 - \beta^2}}
  \end{split}
\end{equation}
where $\alpha^2 - \beta^2 \leq 0$ with equality when ....
\\
The stationary point is non-degenerate so should be able to 
re-write in quadratic form via a change of variables
\\
The frequency depends on the ration $\delta/\epsilon$
%Unlike a numerical scheme we can not arbitrarily improve the accuracy of 
%an asymptotic expansion 
Change of variables...?
$f(\xi)$ is not smoothed but can be smoothed...?
%\textcolor{red}{you might also want to consider the higher order corrections 
%and see if you can approximate the moments too with the same principle...?}
\begin{equation}
  \begin{split}
    &\int_{\mathbb{R}} \text{sgn}(\xi_1) 
    \Theta(\xi_1^2 - 4\delta)\left( 1 + \frac{\xi_1}{\nu(\xi_1)} \right)
    g(\xi_1, \nu(\xi_1))
    \times 
    \\
    &
    \hspace{1cm}
    \exp{\left[-\frac{1}{2\epsilon} 
        \left( \alpha \tilde{a}_1^2 + \overline{\alpha} a_1^2
        + \beta \tilde{a}_1 + \overline{\beta} a_1 + 2q_c|\xi_1 - \nu(\xi_1)|
      \right) 
    \right]}  d \xi_1
  \end{split}
\end{equation}
%%%%%%%%%%%%%%%%%%%%%%%%%
%
%
%     ONE DIMENSION - MONTE CARLO INTEGRATION  
%
%
%%%%%%%%%%%%%%%%%%%%%%%%
\newpage
%%%%%%%%
%
%
%       ONE DIMENSIONAL CASE
%
%
%%%%%%%%
%As discussed in Chapter ..., the solution to the molecular
%schrodinger equation can be expressed in the basis of Hagedorn wavepackets as
%\begin{align}
%  \psi(x,t) = e^{-iS(t)/\epsilon} \sum_{k \in \mathbb{N}^d} c_k(t) \varphi_k^\epsilon[\Pi(t)](x)
%  \label{hagedorn:hagedorn_wavepackets:linear_combination:space}
%\end{align}
%where the equality holds in the $L^2$ norm (?)
%The property of Hagedorn wavepackets with respect to the Fourier Transform
%\textcolor{red}{to verify one can interchange integration and summation
%when going from a position to a momentum representation when expressed as
%an infinite sum... dominated convergence theorem...?}
%yields a representation in a different (w.r.t. the parameter set) basis set in momentum space
%which is given by
%\begin{align}
%  \hat{\psi}(\xi,t) = e^{-iS(t)/\epsilon}e^{-ip \cdot q / \epsilon} \sum_{k \in \mathbb{N}^d}
%  (-i)^{|k|} c_k(t) \varphi_k^\epsilon[\hat{\Pi}(t)](\xi)
%  \label{hagedorn:hagedorn_wavepackets:linear_combination:momentum}
%\end{align}
%where $\hat{\Pi}(t) = (p(t), -q(t), P(t), -Q(t))$ and $|k| = \sum k_j$ (check this)
%\\
%The following holds true for these Hagedorn wavepackets
%\begin{align}
%  \mathcal{F}^{\epsilon} \varphi_k^\epsilon[\Pi](\xi)
%  = (-i)^{|k|}e^{-ip \cdot q / \epsilon} \varphi_k^{\epsilon}[\hat{\Pi}](\xi)
%\end{align}
%The Hagedorn wavepackets satisfy the following recurrence relation in position space
%while in momentum space \textcolor{red}{The
%following should hold true since they still form a basis in momentum space...?}
%\begin{align}
%  Q \left(\sqrt{k_j + 1 } \mathcal{F}^{\epsilon}\varphi^\epsilon_{k + \langle j \rangle} \right)_{j=1}^d
%  =
%  \sqrt{\frac{2}{\epsilon}} (x - q) \mathcal{F}^{\epsilon}\varphi_k^\epsilon(x) -
%  \bar{Q}\left(\sqrt{k_j} \mathcal{F}^{\epsilon}\varphi_{k - \langle j \rangle } \right)_{j=1}^d
%\end{align}
%with $\varphi_0^\epsilon$ given in equation
%\eqref{hagedorn:hagedorn_wavepackets:complex_gaussian} and $\langle j  \rangle$ denoting the
%canonical $d$-dimensional unit vector. The $\varphi_k$ is a polynomial of degree $\sum k_j$
%multiplied with a $d$-dimensional gaussian and we can re-write $\varphi_k^{\epsilon}$
%in the following form for convenience
%\begin{align}
%  \varphi^{\epsilon}_k(x) = p_k(x) \varphi_0^{\epsilon}
%\end{align}
%and similarly in momentum space
%\\
%\\
\end{document}
